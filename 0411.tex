\documentclass{article}
\usepackage[utf8]{inputenc}
\usepackage{enumitem}

\usepackage{seqsplit}
\usepackage{physics}
\usepackage{graphics}

\title{Homework 33}
\author{Austin Frownfelter \and Matthew Bialecki}
\date{April 11, 2018}


\begin{document}

\maketitle

\section{Problem 61}

There are two steps to the proof that NP=L-PCP($\log n$).

\medskip
\noindent First, L-PCP($\log n$) $\subseteq$ NP.

This is trivial.  For a language in L-PCP($\log n$), we can construct an algorithm which instead of randomly checking with $\log n$ random bits, it nondeterministically selects the advice which causes it to accept.

\bigskip
\noindent Then, NP $\subseteq$ L-PCP($\log n$).

To do this, we pick a complete language in NP and show it is also in L-PCP($\log n$).  Let's choose 3SAT.  If L $\in$ 3SAT, then if $x\in $ L all clauses are satisfied, and if $x \notin$ L there exists a clause which is not satisfied.

\medskip
$x\in L \rightarrow \exists y P[M(x,y) accepts] = 1$.

This is trivial.  If $x \in L$, then no matter what clauses M selects, there is a satisfying assignment to those variables.

\medskip
$x\notin L \rightarrow \forall y P[M(x,y) accepts] < 0.5$

Our verifier M takes book advice y, which is a string of variable assignments on x.  We can construct the book such that its entry is a set of copies of the variable assignments, since M has only one-way access to y.  To limit the tape size of M, y will be $n$ copies of the variables.  M can track which "run" it is currently on, then move to an offset which represents the specific clause in the equation, which is selected by $\log n$ random coin flips.  In each test, it chooses a "permutation" of these coin flips, which is just a circulation of a single bit each time.  Thus, M uses $2 \log n$ space.

In this log space, M is able to check a linear number of clauses.  There is a potentially large probability of selecting a clause that would be true even though the formula is not.  Over a linear number of checks, the probability M selects only satisfied clauses shrinks exponentially.

\end{document}
























