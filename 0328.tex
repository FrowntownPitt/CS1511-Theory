\documentclass{article}
\usepackage[utf8]{inputenc}
\usepackage{enumitem}

\usepackage{physics}
\usepackage{graphics}

\title{Homework 27}
\author{Austin Frownfelter \and Matthew Bialecki}
\date{March 28, 2018}


\begin{document}

\maketitle

\section{Problem 49}

$\ket{\mathbf{v}} = \alpha_{00}\ket{00} + \alpha_{10}\ket{10} + \alpha_{01}\ket{01} + \alpha_{11}\ket{11}$

\subsection{Part a}

$P(v=00) = (\alpha_{00}) ^2$

\noindent$P(v=10) = (\alpha_{10}) ^2$

\noindent$P(v=01) = (\alpha_{01}) ^2$

\noindent$P(v=11) = (\alpha_{11}) ^2$

\medskip
\noindent In the case where the probability of a qubit being 0 or 1 is uniform, all probabilities are $\frac14$


\subsection{Part b}
$\ket{v_0v_1} = \sqrt{(\alpha_{00})^2 + (\alpha_{01})^2}\ket{0v_1} +  \sqrt{(\alpha_{10})^2 + (\alpha_{11})^2}\ket{1v_1} = $

$\sqrt{(\alpha_{00})^2 + (\alpha_{01})^2} (\frac{\alpha_{00}}{\sqrt{(\alpha_{00})^2 + (\alpha_{01})^2}}\ket{00} +  \frac{\alpha_{01}}{\sqrt{(\alpha_{00})^2 + (\alpha_{01})^2}}\ket{01}) + $

$\sqrt{(\alpha_{10})^2 + (\alpha_{11})^2} (\frac{\alpha_{10}}{\sqrt{(\alpha_{10})^2 + (\alpha_{11})^2}}\ket{10} +  \frac{\alpha_{11}}{\sqrt{(\alpha_{10})^2 + (\alpha_{11})^2}}\ket{11})$

\medskip
$= \alpha_{00}\ket{00} + \alpha_{10}\ket{10} + \alpha_{01}\ket{01} + \alpha_{11}\ket{11}$

\bigskip
\noindent Thus,

$P(v=00) = (\alpha_{00}) ^2$

$P(v=10) = (\alpha_{10}) ^2$

$P(v=01) = (\alpha_{01}) ^2$

$P(v=11) = (\alpha_{11}) ^2$



\subsection{Part c}
$\ket{v_0v_1} = \sqrt{(\alpha_{00})^2 + (\alpha_{10})^2}\ket{v_00} +  \sqrt{(\alpha_{01})^2 + (\alpha_{11})^2}\ket{v_01} = $

$\sqrt{(\alpha_{00})^2 + (\alpha_{10})^2} (\frac{\alpha_{00}}{\sqrt{(\alpha_{00})^2 + (\alpha_{10})^2}}\ket{00} +  \frac{\alpha_{10}}{\sqrt{(\alpha_{00})^2 + (\alpha_{10})^2}}\ket{10}) + $

$\sqrt{(\alpha_{01})^2 + (\alpha_{11})^2} (\frac{\alpha_{01}}{\sqrt{(\alpha_{01})^2 + (\alpha_{11})^2}}\ket{01} +  \frac{\alpha_{11}}{\sqrt{(\alpha_{01})^2 + (\alpha_{11})^2}}\ket{11})$

\medskip
$= \alpha_{00}\ket{00} + \alpha_{10}\ket{10} + \alpha_{01}\ket{01} + \alpha_{11}\ket{11}$

\bigskip
\noindent Thus,

$P(v=00) = (\alpha_{00}) ^2$

$P(v=10) = (\alpha_{10}) ^2$

$P(v=01) = (\alpha_{01}) ^2$

$P(v=11) = (\alpha_{11}) ^2$

\section{Problem 50}

\subsection{Part a}

$x=y=1$

\medskip
$\ket{ij} = \frac1{\sqrt2}\ket{00} + \frac1{\sqrt2}\ket{11}$  - This is because the particles are entangled.

\medskip
Alice rotates by $\frac\pi8$.  This can be represented with a transformation matrix.

\medskip
After rotating, $\ket{ij} = \frac1{\sqrt2} ( \cos\frac\pi8 \ket{00} + \sin\frac\pi8\ket{10}) + \frac1{\sqrt2} ( \cos\frac{5\pi}8 \ket{01} + \sin\frac{5\pi}8\ket{11})$.  Each amplitude is the corresponding value in the transformation.  Note that the probabilities (square of amplitudes) sum to 1.

\medskip
Bob rotates by $-\frac\pi8$.  This can also be represented with a transformation matrix.

\medskip
After rotating, $\ket{ij} = \frac1{\sqrt2} ( \cos\frac\pi8 \ket{00} - \sin\frac\pi8\ket{10}) + \frac1{\sqrt2} ( \cos\frac{3\pi}8 \ket{01} + \sin\frac{3\pi}8\ket{11})$.  Each amplitude is the corresponding value in the transformation.  Note that the probabilities (square of amplitudes) sum to 1.


\bigskip
Alice measures her entangled particle.  

\medskip
$P(0) = (\frac1{\sqrt2} \cos\frac\pi8)^2 + (\frac1{\sqrt2} \cos\frac{5\pi}8)^2 = \frac12$.  This is the sum of the probabilities Alice's qubit is 0, where Bob's is 0 or 1.

$P(1) = (\frac1{\sqrt2} \sin\frac\pi8)^2 + (\frac1{\sqrt2} \sin\frac{5\pi}8)^2 = \frac12$.  This is the sum of the probabilities Alice's qubit is 1, where Bob's is 0 or 1.

\bigskip
Bob measures his entangled particle.

\medskip

When Alice reads a 0, Bob's qubit is in the following state:

$\ket{0b}=\frac{\frac1{\sqrt2} \cos \frac\pi8}{ \sqrt{(\frac1{\sqrt{2}}\cos(\frac\pi8))^2+(\frac1{\sqrt{2}}\cos(\frac{3\pi}8))^2}} \ket{00} + $
$\frac{\frac1{\sqrt2} \cos \frac{3\pi}8}{ \sqrt{(\frac1{\sqrt{2}}\cos(\frac\pi8))^2+(\frac1{\sqrt{2}}\cos(\frac{3\pi}8))^2}} \ket{01}$

\medskip
When Alice reads a 1, Bob's qubit is in the following state:

$\ket{0b}=\frac{\frac1{\sqrt2} \sin -\frac\pi8}{ \sqrt{(\frac1{\sqrt{2}}\sin(\frac\pi8))^2+(\frac1{\sqrt{2}}\sin(\frac{3\pi}8))^2}} \ket{00} + $
$\frac{\frac1{\sqrt2} \sin \frac{3\pi}8}{ \sqrt{(\frac1{\sqrt{2}}\sin(\frac\pi8))^2+(\frac1{\sqrt{2}}\sin(\frac{3\pi}8))^2}} \ket{01}$

\medskip
Thus, 

$P(0|$Alice $=0)= \frac{\frac{1}{2} \cos^2\frac\pi8}{\frac{1}{2}} = \cos^2\frac\pi8 \approx 0.85355$

$P(0|$Alice $=1)= \frac{\frac{1}{2} \sin^2\frac\pi8}{\frac{1}{2}} = \sin^2\frac\pi8 \approx 0.14645$

$P(1|$Alice $=0)= \frac{\frac{1}{2} \cos^2\frac{3\pi}8}{\frac{1}{2}} = \cos^2\frac{3\pi}8 \approx 0.14645$

$P(1|$Alice $=1)= \frac{\frac{1}{2} \sin^2\frac{3\pi}8}{\frac{1}{2}} = \sin^2\frac{3\pi}8 \approx 0.85355$

\medskip
Thus, the resulting probability Alice and Bob win when $x=y=1 \approx  0.14645$ 

\subsection{Part b}

The protocols appear to be the same.  Therefore, the probabilities should be equal.

\medskip
\noindent When $x=y=0$ the probability they win is 1.

\medskip
\noindent When $x\neq y$, the probability they win $\approx  0.85355$ (when $a = b$).  This is taken from the calculation above, which is equivalent to the book.

\medskip
\noindent When $x=y=1$, the probability they win is $\approx  0.14645$ (when $a\neq b$).  This is taken from the calculation above, which should be equivalent to the book.

\bigskip
The resulting probability they win is $\frac14*1 + \frac12*0.8535 + \frac 14 * 0.1465 \approx 0.71$

This is less than the probability in the book, which means the math done in part a for us was incomplete.  The 0.1465 should be 0.5.  However, the probability they win in the lecture's protocol is equal to the book's protocol, or 0.8.

\end{document}
























