\documentclass{article}
\usepackage[utf8]{inputenc}
\usepackage{enumitem}

\title{Homework 9}
\author{Austin Frownfelter \and Matthew Bialecki}
\date{February 2, 2018}


\begin{document}

\maketitle

\section{Problem 13}
\subsection{$M\rightarrow N$}
Let $N=M$, $b=1$.
\\Since M accepts x if $x\in L$ in time $T(|x|)$, N will accept x in time $1*T(|x|)$.

\subsection{$N\rightarrow M$}
To construct M from N, increase the alphabet size of M by a factor of b.  Reconstruct M to be able to accept these larger alphabet symbols.

Construct a mapping of permutations of symbols of length 1 to b in N to the alphabet in M.

Begin at the start state of N

Search using a DFS to a depth of $k*b$

Pause at this point and create a transition from the parent state ($(k-1)*b$) to the current state using the single alphabet symbol mapped by the path used to get there.

If the DFS gets to a depth $\neq k*b$ and cannot go farther, create a transition from the parent state to the current state using the single alphabet symbol mapped by the path used to get there.

M is a "sped up" version of N, since N's transitions are traversed $b$-at-a-time.

Since there exists a mapping of $M\rightarrow N$ and $N\rightarrow M$, the two definitions are equivalent.

\end{document}
