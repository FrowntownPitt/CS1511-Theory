\documentclass{article}
\usepackage[utf8]{inputenc}
\usepackage{enumitem}

\title{Homework 16}
\author{Austin Frownfelter \and Matthew Bialecki}
\date{February 19, 2018}


\begin{document}

\maketitle

\section{Problem 25}
\begin{quote}
Show that if a sparse language is NP-complete, then P=NP.
\end{quote}

Let S be an NP-complete sparse language.  Since S is NP-complete, there exists a polytime reduction $f$ from LSAT to S.

Define LSAT to be a variant of SAT where $(\phi, x)\in$ LSAT iff $\phi \in$ SAT $\land$ $x\in S$.

Since f is polytime, $|f(x)| \leq |x|^k$ for some k.

The input of LSAT $(\phi, x)$ is converted by $f(x)$ to be the input of S.  Since f is polytime, the output of f $\leq p(n)$ where $n=|x|$ and $p(n)$ is a polynomial upper bound.  Assign q(x) to the the size of the intersection of the set S and the set of strings $\{0,1\}^{\leq p(n)}$.  q(x) is polynomial by sparseness of f.

Imagine an algorithm for LSAT as a tree.  The root's children are decided by a variable assignment of the remaining $\phi$.  If the number of children (elements in the next row of the tree) is greater than q(n), then prune the branches until the row is q(n) wide using the following policy:
\begin{enumerate}[label=]
\item if any branches yield the same string, keep only 1 of them
\item if there are more than q(x) unique strings, choose q(x) to keep
\end{enumerate}

Repeat this process until all variables have been assigned.  Accept if there exists a child in the final row of the tree that satisfies the formula. 

Since f is polytime, deciding the children takes polytime.  Since there are n levels (n variables), and the width of the tree (number of paths) is at most q(n), the runtime is $nq(n)$ which is polynomial.

\bigskip
Therefore, if there is an NP-complete sparse language, NP-complete languages can be reduced to it to run them in polynomial time, implying P=NP.

(Note: the following websites were used to understand this problem
\begin{enumerate}
\item https://math.stackexchange.com/questions/235162/if-an-unary-language-exists-in-npc-then-p-np
\item http://www.cs.umd.edu/~jkatz/complexity/f05/lecture6.pdf)
\end{enumerate}

\section{Problem 26}
\begin{quote}
Show that ZPP = RP $\cap$ coRP
\end{quote}

ZPP = RP $\cap$ CoRP

\bigskip
ZPP $\supseteq$ RP $\cap$ CoRP

Let language L be in RP and CoRP:

There exists a TM A that accepts $x$ with probability $\geq$ 0.6 if $x \in L$

There exists a TM B that rejects $x$ with probability $\geq$ 0.6 if $x \notin L$

\medskip
TM C on input x:
\begin{enumerate}[label=]
\item for k iterations:
	\begin{enumerate}[label=]
	\item Run A on x.  If accept, accept
	\item Run B on x.  If reject, reject
	\end{enumerate}
\item reject
\end{enumerate}

The probability C accepts if $x \in L$ = $1-P[A(x)$ is wrong$]^k$, rejects if $x \notin L$ = $1-P[B(x)$ is wrong$]^k$.  As $k \rightarrow \infty$, the probability C is wrong exponentially shrinks to 0.  Thus, the probability of C being correct is 1 if $k = \infty$.

The expected runtime for C is polynomial.  Therefore, ZPP contains RP and CoRP.

\bigskip
ZPP $\subseteq$ RP $\cap$ CoRP

ZPP $\subseteq$ RP.

Run C on input x for at least double its expected runtime.  If it hasn't halted in this time, reject.

By Markov's inequality, the probability C yields a result in this time is at least $\frac12$.  If it hasn't decided by then, reject, since the probability it is wrong when $x\in L$ is less than $\frac12$, matching the definition of RP.


\medskip
ZPP $\subseteq$ CoRP.

Run C on input x for at least double its expected runtime.  If it hasn't halted in this time, accept.

By Markov's inequality, the probability C yields a result in this time is at least $\frac12$.  If it hasn't decided by then, accept, since the probability it is wrong when $x\notin L$ is less than $\frac12$, matching the definition of CoRP.

\bigskip
Therefore, ZPP $\subseteq$ RP $\cap$ CoRP.

\end{document}
























