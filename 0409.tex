\documentclass{article}
\usepackage[utf8]{inputenc}
\usepackage{enumitem}

\usepackage{seqsplit}
\usepackage{physics}
\usepackage{graphics}

\title{Homework 32}
\author{Austin Frownfelter \and Matthew Bialecki}
\date{April 9, 2018}


\begin{document}

\maketitle

\section{Problem 60}

\subsection{Part a}

$u=101$


\subsection{Part b}

Refer to the attached python program.  It either prints "Failed" if the protocol catches the verifier, or the binary and hexadecimal representation of the book entry.


\subsection{Part c}
The program prints the binary and hexadecimal representations of the book.  It uses the $``|"$ operator to separate the $WH(u) and WH(u \otimes u)$ (for readability)

$\seqsplit{% 
01011010|01011010010110100101101001011010010110100101101001011010010110101010010110100101101001011010010110100101101001011010010110100101010110100101101001011010010110100101101001011010010110100101101010100101101001011010010110100101101001011010010110100101101001011010010110100101101001011010010110100101101001011010010110100101010110100101101001011010010110100101101001011010010110100101101010100101101001011010010110100101101001011010010110100101101001010101101001011010010110100101101001011010010110100101101001011010}$

$\seqsplit{%
0x5a|0x5a5a5a5a5a5a5a5aa5a5a5a5a5a5a5a55a5a5a5a5a5a5a5aa5a5a5a5a5a5a5a5a5a5a5a5a5a5a5a55a5a5a5a5a5a5a5aa5a5a5a5a5a5a5a55a5a5a5a5a5a5a5a}$


\subsection{Part d}

The first 8 bits ($2^3$) represent the values of the inner product of the input and $i$, where $i = \{0\ldots7\}$.  The last 512 bits ($2^{3^2}=2^9$) represent the inner product of $u \otimes u$ and $x$, where $x = \{0\ldots511\}$.

$x$ is ordered such that each $x_i$ represents $u_1u_1+u_1u_2+u_1u_3 + u_2u_1+u_2u_2+u_2u_3 + u_3u_1+u_3u_2+u_3u_3$, where each $1$ bit means that part of the equation is included ($010000000 \equiv u_1u_2$).

With the above notation, the equation $u_1u_2+u_2u_2+u_3u_3$ has the 2nd, 5th, and 9th bits in $x_i=1$, with the rest being 0.  This equation becomes the string $010010001$.  Since it makes sense to order the last 512 bits numerically, and this is the (decimal) number 145, the 153rd bit ($145+8$) in the resulting book entry represents the bit of this equation.

\end{document}
























