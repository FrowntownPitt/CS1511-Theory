\documentclass{article}
\usepackage[utf8]{inputenc}
\usepackage{enumitem}

\title{Homework 15}
\author{Austin Frownfelter \and Matthew Bialecki}
\date{February 16, 2018}


\begin{document}

\maketitle

\section{Problem 23}
\begin{quote}
``Describe a decidable language in P/poly that is not in P''
\end{quote}


Assume $L \notin EXP$.

$L'=\{1^m | m\in L\}$

There exists a family of circuits which decides $L'$ in polynomial gates ($L'\in P/poly$)

If $L'$ runs in polynomial time on input $1^m$ (relative to $m$) then $L$ runs in exponential time on input $m$ (relative to $n = \log m$).

This contradicts with the assumption, which means L must be contained in $EXP$.  Therefore, this language is in P/poly but not in P.


\section {Problem 24}
\subsection{Part a}
\begin{quote}
Show for every $k > 0$ that PH contains languages whose circuit complexity is $\Omega(n^k)$
\end{quote}

For any given k, construct a TM that decides a $\Sigma_k^p$ language as such: The input is a quantified boolean formula with k input variables.

There exists a circuit which "simulates" this TM.

This circuit's complexity is greater than $n^k$.

Therefore, for all k, there exists a language which is decided by a family of circuits where $|C_n|\geq n^k$

\subsection{Part b}
\begin{quote}
Solve question 6.5 with PH replaced by $\Sigma_{2}^p$ (if your solution didn't already do this).
\end{quote}

By the logic in the previous problem, there existed a circuit which decided a $\Sigma_2^p$ language in complexity $\geq n^k$.

\subsection{Part c}
\begin{quote}
Show that if P = NP, then there is a language in EXP that requires circuits of size $\frac{2^n}n$.
\end{quote}

If P = NP then the Polynomial Time Hierarchy would collapse, having all $\Sigma_k^p$ and $\Pi_k^p$ problems be solved in time of a $\Sigma_1^p$ problem.  Since EXP must be a strict superset of PH (per the Time Hierarchy Theorem), the circuit lower boundary of EXP would decrease to $\frac{2^n}n$.

\end{document}












