\documentclass{article}
\usepackage[utf8]{inputenc}
\usepackage{enumitem}

\title{Homework 11}
\author{Austin Frownfelter \and Matthew Bialecki}
\date{February 7, 2018}


\begin{document}

\maketitle

\section{Problem 16}
\begin{quote}
``Show that if there is a log space Turing machine S that accepts B, then there is a log space Turing Machine U that accepts A.''
\end{quote}

There is a TM T which its output is in B iff $x \in A$.  If there is a log space TM S that accepts B, then we can construct a TM U that accepts A:

\bigskip
U inputs x, and it must run T on x to get its output, which is the input to S to decide if it is in B.  If it is in B, then x is in A.

We can simulate T on x until it outputs its first symbol, which is then fed into S.  T maintains its work tape, along with its current output, on the work tape of U.  Once T has output its first symbol, S is simulated until it requires another output symbol of T.  This required symbol's index $i$ is stored on the work tape of U, along with the work of S.  U then simulates T until it reaches its $i$th output, which is stored on the work tape.  S is then simulated from where it left off until it requires another output of T, where the process is then repeated.  U continues this process until S either accepts or rejects, which U will then accept or reject, respectively.  

U has a work tape which stores the individual work tapes of S and T.  It also has to store the state of S when it runs T such that it can continue S later, along with the $i$th output of T.  In total, the work tape of U uses $2*log(x)+c$ where c is the constants which are the state of S and the $i$th output of T.

Therefore, if there is a log space TM S that accepts B, there exists a log space TM U that accepts A.

\section{Problem 17}
\subsection{Part a}
\begin{quote}
``Define a language C, and show that C is complete for EXPSPACE under polynomial time reductions.''
\end{quote}

$C = \{(M,I,2^{n^{k}})$ $|$ $M$ accepts $I$ in space $2^{n^{k}}\}$

\medskip
$C$ is complete for EXPSPACE under polynomial time reductions.
\begin{enumerate}[label=\arabic*)]
    \item $C \in$ EXPSPACE
    
    There exists a TM $N$(TM $M$, input $I$, $2^{n^{k}}$) that decides $C$:
    \begin{enumerate}[label=]
        \item Run $M$ on $I$ for $2^{n^{k}}$ steps
        \item If $M$ accepted or rejected, accept or reject respectively
        \item Otherwise, reject
    \end{enumerate}

    $N$ decides $C$ in EXPSPACE, therefore $C\in$ EXPSPACE.
    
    
    \item $\forall L \in$ EXPSPACE, $L \leq_{p} C$
    
    Let $L \in$ EXPSPACE
    
    $\exists$ TM $M$, integer $k$ $|$ $M$ accepts $x$ iff $x \in L$, $M$ on $x$ uses space $\leq 2^{|x|^{k}}$.
    
    $\exists$ TM $A$ which constructs input $(M,I,2^{|x|^{k}})$ and runs a decider for $C$ on it.  If the decider accepts, $A$ accepts; otherwise it rejects.
    
    $M$ and $I$ can be constructed in polynomial time. $2^{n^{k}}$ can be encoded using polynomial space in polynomial time.  
\end{enumerate}

\subsection{Part b}
\begin{quote}
``Define a language C, and show that C is complete for EXPSPACE under polynomial time reductions.''
\end{quote}

$C = \{(M,I,c^{n^{k}})$ $|$ $M$ accepts $I$ in space $c^{n^{k}}\}$

\medskip
$C$ is complete for EXPSPACE under polynomial time reductions.
\begin{enumerate}[label=\arabic*)]
    \item $C \in$ EXPSPACE
    
    There exists a TM $N$(TM $M$, input $I$, $c^{n^{k}}$) that decides $C$:
    \begin{enumerate}[label=]
        \item Run $M$ on $I$ for $c^{n^{k}}$ steps
        \item If $M$ accepted or rejected, accept or reject respectively
        \item Otherwise, reject
    \end{enumerate}

    $N$ decides $C$ in EXPSPACE, therefore $C\in$ EXPSPACE.
    
    
    \item $\forall L \in$ EXPSPACE, $L \leq_{p} C$
    
    Let $L \in$ EXPSPACE
    
    $\exists$ TM $M$, integer $k$ $|$ $M$ accepts $x$ iff $x \in L$, $M$ on $x$ uses space $\leq c^{|x|^{k}}$.
    
    $\exists$ TM $A$ which constructs input $(M,I,c^{|x|^{k}})$ and runs a decider for $C$ on it.  If the decider accepts, $A$ accepts; otherwise it rejects.
    
    $M$ and $I$ can be constructed in polynomial time. $c^{n^{k}}$ can be encoded using polynomial space in polynomial time.
\end{enumerate}

\end{document}












