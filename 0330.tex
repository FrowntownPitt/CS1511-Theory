\documentclass{article}
\usepackage[utf8]{inputenc}
\usepackage{enumitem}

\usepackage{physics}
\usepackage{graphics}

\title{Homework 28}
\author{Austin Frownfelter \and Matthew Bialecki}
\date{March 30, 2018}


\begin{document}

\maketitle

\section{Problem 51}

In the final observation in Simon's algorithm, we observe $y$ in the first $n$ bits iff $a \odot y \mod 2 = 0$.  When $a = 0$, the inner product of $a$ and $y$ will always be zero.  Thus, we will always observe $y$ in the first $n$ bits when $a=0$.

In the case where $x=0^n$ and $a=0^n$, $y=0^n$ and we will learn no bits of $a$ ($0=0$).  In all other cases, we discover the sum of $a_i$ bits where $y_i=1$ is equal to zero.  If the algorithm is run $2n$ times, there is an extremely high likelihood that we get $n$ independent equations, allowing us to solve for all bits in $a$.  

Therefore, if $a=0^n$, Simon's algorithm still works.

\section{Problem 52}

\subsection{Part a}

We applied the Kroenecker product ($\otimes$) on $H_1$ and the basis (denoted as $I$):

$H_1 \otimes I = \frac1{\sqrt2}(\begin{smallmatrix} 1&1\\1&-1 \end{smallmatrix}) \otimes (\begin{smallmatrix} 1&0\\0&1 \end{smallmatrix})=$

\[
\frac1{\sqrt2}
\begin{bmatrix}
1 & 0 & 1 & 0\\
0 & 1 & 0 & 1\\
1 & 0 & -1 & 0\\
0 & 1 & 0 & -1
\end{bmatrix}
\]

\subsection{Part b}

In Part A we did $H_1\otimes I$ because the $H_1$ is applied to the first qubit, and the second qubit remains the same.  Thus, we took the Kroenecker product ($\otimes$) on the basis $I$ and $H_1$:

$I \otimes H_1 = (\begin{smallmatrix} 1&0\\0&1 \end{smallmatrix}) \otimes \frac1{\sqrt2}(\begin{smallmatrix} 1&1\\1&-1 \end{smallmatrix})=$

\[
\frac1{\sqrt2}
\begin{bmatrix}
1 & 1 & 0 & 0\\
1 & -1 & 0 & 0\\
0 & 0 & 1 & 1\\
0 & 0 & 1 & -1
\end{bmatrix}
\]


\subsection{Part c}

$
\frac1{\sqrt2}
\begin{bmatrix}
1 & 0 & 1 & 0\\
0 & 1 & 0 & 1\\
1 & 0 & -1 & 0\\
0 & 1 & 0 & -1
\end{bmatrix}
\cdot
\frac1{\sqrt2}
\begin{bmatrix}
1 & 1 & 0 & 0\\
1 & -1 & 0 & 0\\
0 & 0 & 1 & 1\\
0 & 0 & 1 & -1
\end{bmatrix}
=
\frac12
\begin{bmatrix}
1 & 1 & 1 & 1\\
1 & -1 & 1 & -1\\
1 & 1 & -1 & -1\\
1 & -1 & -1 & 1
\end{bmatrix}$


\subsection{Part d}


$H_2(a\ket{00} + b\ket{01} + c\ket{10} + d\ket{11}) = $

\medskip
$
\frac1{\sqrt2}
\begin{bmatrix}
1 & 1 & 1 & 1\\
1 & -1 & 1 & -1\\
1 & 1 & -1 & -1\\
1 & -1 & -1 & 1
\end{bmatrix}
\cdot
\frac1{\sqrt2}
\begin{bmatrix}
a & 0 & 0 & 0\\
0 & b & 0 & 0\\
0 & 0 & c & 0\\
0 & 0 & 0 & d
\end{bmatrix}
=
\frac12
\begin{bmatrix}
a & b & c & d\\
a & -b & c & -d\\
a & b & -c & -d\\
a & -b & -c & d
\end{bmatrix} = $
\medskip

\smallskip
$a\frac12(\ket{00}+\ket{01}+\ket{10}+\ket{11}) + $

\smallskip
$b\frac12(\ket{00}- \ket{01}+\ket{10}- \ket{11}) + $

\smallskip
$c\frac12(\ket{00}+\ket{01}- \ket{10}- \ket{11}) + $

\smallskip
$d\frac12(\ket{00}- \ket{01}- \ket{10}+\ket{11}) = $

\smallskip
$\frac12[(a+b+c+d)\ket{00} + (a-b+c-d)\ket{01} + (a+b-c-d)\ket{10} + (a-b-c+d)\ket{11})]$


\subsection{Part e}

$(H_1\otimes I)\cdot(a\ket{00} + b\ket{01} + c\ket{10} + d\ket{11}) = $

\medskip
$
\frac1{\sqrt2}
\begin{bmatrix}
1 & 0 & 1 & 0\\
0 & 1 & 0 & 1\\
1 & 0 & -1 & 0\\
0 & 1 & 0 & -1
\end{bmatrix}
\cdot
\frac12
\begin{bmatrix}
a & 0 & 0 & 0\\
0 & b & 0 & 0\\
0 & 0 & c & 0\\
0 & 0 & 0 & d
\end{bmatrix}
=
\frac1{\sqrt2}
\begin{bmatrix}
a & 0 & c & 0\\
0 & b & 0 & d\\
a & 0 & -c & 0\\
0 & b & 0 & -d
\end{bmatrix} = $
\medskip

\smallskip
$a\frac1{\sqrt2}(\ket{00}+\ket{10}) + $

\smallskip
$b\frac1{\sqrt2}(\ket{01}+\ket{11}) + $

\smallskip
$c\frac1{\sqrt2}(\ket{00}-\ket{10}) + $

\smallskip
$d\frac1{\sqrt2}(\ket{01}-\ket{11}) = $

\medskip
$\frac1{\sqrt2}((a+c)\ket{00}+ (b+d)\ket{01} + (a-c)\ket{10} + (b-d)\ket{11})$


\subsection{Part f}
$
\frac1{\sqrt2}
\begin{bmatrix}
1 & 1 & 0 & 0\\
1 & -1 & 0 & 0\\
0 & 0 & 1 & 1\\
0 & 0 & 1 & -1
\end{bmatrix}
\cdot
\frac1{\sqrt2}
\begin{bmatrix}
a & 0 & 0 & 0\\
0 & b & 0 & 0\\
0 & 0 & c & 0\\
0 & 0 & 0 & d
\end{bmatrix}
=
\frac12
\begin{bmatrix}
(a+c) & (b+d) & 0 & 0\\
(a+c) & -(b+d) & 0 & 0\\
0 & 0 & (a-c) & (b-d)\\
0 & 0 & (a-c) & -(b-d)
\end{bmatrix} = $

\smallskip
$\frac12[ (a+c)(\ket{00}+\ket{01}) +$

\smallskip
$(b+d)(\ket{00}-\ket{01}) +$

\smallskip
$(a-c)(\ket{10}+\ket{11}) + $

\smallskip
$(b-d)(\ket{10}-\ket{11})] = $

\medskip
$\frac12[(a+b+c+d)\ket{00} + (a-b+c-d)\ket{01} + (a+b-c-d)\ket{10} + (a-b-c+d)\ket{11})]$



\end{document}
























